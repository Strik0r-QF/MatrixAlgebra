\documentclass[utf-8, 10pt]{article}

\usepackage{ctex}
\input{LaTeX_tem/geometry_settings.tex}
\usepackage{titlesec} % 定义标题样式

% 定义 section 标题格式
\titleformat{\section}[hang]{\heiti\centering\large\bfseries}{\thesection}{1em}{}

% 定义 subsection 标题格式
\titleformat{\subsection}[hang]{\heiti\bfseries}{\textbf{\thesubsection}}{1em}{}

% 定义 subsubsection 标题格式
\titleformat{\subsubsection}[hang]{\kaishu}{\quad\quad\thesubsubsection\,\,}{0em}{}
\input{LaTeX_tem/mdframed_settings.tex}
\input{LaTeX_tem/listings_settings.tex}
\usepackage{caption, subcaption}
\usepackage{longtable, diagbox, booktabs}
\usepackage{float, graphicx}
\usepackage{amsthm, amssymb, amsmath, mathrsfs, mhchem, siunitx, pgfplots}
\usepackage{tikz, circuitikz, tikz-cd, tikz-3dplot}
\usetikzlibrary{decorations.markings, angles, quotes}
\usepackage{tasks, enumitem}
\usepackage{hyperref}
\hypersetup{hidelinks,
    colorlinks = true,
    allcolors = black,
    pdfstartview = Fit,
    breaklinks = true}
\usepackage[toc]{multitoc}
\usepackage{abstract}
\usepackage{extpfeil}
\usepackage{xcolor}

\everymath{\displaystyle}

\begin{document}
\input{LaTeX_tem/theoremstyles.tex}

\title{\textbf{矩阵代数}

Matrix Algebra}
\author{钱锋\thanks{电子邮件: strik0r.qf@gmail.com
\newline \indent 西北工业大学软件学院, School of Software, Northwestern Polytechnical University, 西安 710072
\newline \indent 保密级别: public}}
\setlength{\columnsep}{2em}
\twocolumn[
    \maketitle
    {\small \tableofcontents}
    \newpage
    \begin{abstract}
        现在还没有摘要
        \begin{keywords}
            
        \end{keywords}
    \end{abstract}
    \vspace*{1em}
]
\saythanks

% 正文从这里开始
\section{矩阵运算}

\textbf{矩阵} (matrix) 是\textbf{二维数组} (2 dimentional array), 
它由 $m \times n$ 个\textbf{元素} (element) 排列成 $m$ \textbf{行} (row), 
$n$ \textbf{列} (coloum) 组成.
矩阵通常用粗体的拉丁字母表示.
矩阵的行数与列数构成的二维向量 $[m, n]$ 
称为矩阵 $\boldsymbol{A}$ 的\textbf{形状} (shape). 记作 $\boldsymbol{A}.\mathrm{shape}$,
形状为 $[m, n]$, 且所有元素取自数域 $\mathbb{K}$ 的矩阵称为
数域 $\mathbb{K}$ 上的 $m \times n$ 矩阵, 这样的矩阵全体组成的集合
记作 $\mathbb{K}^{m \times n}$, 称为 $m \times n$ \textbf{矩阵空间}
(matrix space).

设矩阵 $\boldsymbol{A} \in \mathbb{K}^{m \times n}$, 则 $\boldsymbol{A}$ 的第 $i$
行和第 $j$ 列交叉位置处的元素用 $a_{ij}$ 或 $\boldsymbol{A}[i,j]$ 来表示. 即
\[ \boldsymbol{A} = \begin{bmatrix}
    a_{00} & a_{01} & \cdots & a_{0, n-1} \\
    a_{10} & a_{11} & \cdots & a_{1, n-1} \\ 
    \vdots & \vdots & \ddots & \vdots \\ 
    a_{m-1, 0} & a_{m-1, 0} & \cdots & a_{m-1,n-1}
\end{bmatrix}. \]
为了方便将矩阵方程转变为高级程序设计语言代码, 我们在设计数学符号时
使用了面向对象风格\footnote{
    一般地, 与具体的对象有关的属性和函数采用面向对象风格的记号,
    与具体的对象无关的属性和函数 (即静态的属性和函数) 和多元函数采用传统的函数记号.
}, 并规定矩阵中元素的行索引和列索引均从 $0$ 开始, 而不是从
$1$ 开始.

矩阵 $\boldsymbol{A}$ 的各列是 $\mathbb{R}^m$ 中的向量,
我们用黑体字母 $\boldsymbol{a}_0, 
\boldsymbol{a}_1, \cdots, \boldsymbol{a}_{n-1}$
来表示, 也可以采用高级程序设计语言中常见的记号,
使用 $\boldsymbol{A}[:,0], \boldsymbol{A}[:,1], \cdots, \boldsymbol{A}[:,n-1]$
来表示. 因此我们也可以写作
\[ \boldsymbol{A} = \left[
    \boldsymbol{a}_0,
    \boldsymbol{a}_1, \cdots, \boldsymbol{a}_{n-1}
\right]. \]
其中, $\boldsymbol{A}[i,j] = \boldsymbol{A}[:,j][i]$, 它表示元素 $a_{ij}$ 是第 $j$ 个列向量 
$\boldsymbol{a}_j$ 的第 $i$ 个元素.
在后续的行文中, 除非形如 $a_{ij}$ 的表示法不便于表示 (例如 $i$ 或 $j$ 的表达式
特别复杂), 或者需要将矩阵方程翻译为高级程序设计语言代码, 否则我们一般不采用
符号 $\boldsymbol{A}[i,j]$.

$m \times n$ 矩阵 $\boldsymbol{A}$ 的对角线元素

\clearpage
\section{矩阵的逆}
\section{可逆矩阵的特征}
\section{分块矩阵}
\section{矩阵分解}



\begin{thebibliography}{1}
    \addcontentsline{toc}{section}{参考文献}
    \bibitem{David}
    [美] David C. Lay, [美] Steven R. Lay, [美] Judi J. McDonald.
    线性代数及其应用: 原书第 6 版 = Linear Algebra and Its Application,
    Sixth Edition [M]. 刘深泉等译. 北京: 机械工业出版社, 2023.
\end{thebibliography}

\end{document}